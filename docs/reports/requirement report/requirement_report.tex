\documentclass[11pt]{article}
\usepackage{fontspec}
\usepackage[margin=1in,left=1.5in,includefoot]{geometry}
\usepackage{listings}
\usepackage{amsmath}
\usepackage{color}

\definecolor{dkgreen}{rgb}{0,0.6,0}
\definecolor{gray}{rgb}{0.5,0.5,0.5}
\definecolor{mauve}{rgb}{0.58,0,0.82}

% Graphic
\usepackage{graphicx}
\usepackage{float}

% Hyperlinks
\usepackage{hyperref}

% Header and footer
\usepackage{fancyhdr}
\pagestyle{fancy}
\fancyfoot{}
\fancyhead[LE,RO]{\bfseries\thepage}
\setlength{\headheight}{15pt}

% Quotes character
\usepackage[utf8]{inputenc}

% color python code
\lstset{frame=tb,
  language=Python,
  aboveskip=3mm,
  belowskip=3mm,
  showstringspaces=false,
  columns=flexible,
  basicstyle={\small\ttfamily},
  numbers=none,
  numberstyle=\tiny\color{gray},
  keywordstyle=\color{blue},
  commentstyle=\color{dkgreen},
  stringstyle=\color{mauve},
  breaklines=true,
  breakatwhitespace=true,
  tabsize=3
}

% color links
\usepackage{color}  
\usepackage{hyperref}
\hypersetup{
    colorlinks=true, %set true if you want colored links
    linktoc=all,     %set to all if you want both sections and subsections linked
    linkcolor=black,  %choose some color if you want links to stand out
    urlcolor=blue
}

\begin{document}
\begin{titlepage}
	\begin{center}
\includegraphics[width=0.6\textwidth]{images/bordeaux.png}\\[1cm]


{\large Report Requirement}\\[0.5cm]	
	
	\line(1,0){400}\\[0.2in]
	\huge{\bfseries Deep Learning}\\
	\line(1,0){400}\\[1.5cm]
	
	\noindent	
	
	\begin{minipage}[t]{0.4\textwidth}
		\begin{flushleft} \large
    	\emph{Author:}\\%
    	Manh Tu \textsc{Vu}
		\end{flushleft}
	\end{minipage}
	\begin{minipage}[t]{0.4\textwidth}
  		\begin{flushright} \large
    		\emph{Supervisor:} \\
    		Marie\textsc{Beurton-Aimar}
  		\end{flushright}
	\end{minipage}
	
	\vfill

% Bottom of the page
{\large \today}
	\end{center}
\end{titlepage}

% Front matter
\pagenumbering{arabic}

% Table content
\tableofcontents
\thispagestyle{empty}
\clearpage

% List of figures 
%\listoffigures
%\clearpage

\section{Topic}
In this project, we use Deep Learning method to automatic classify images from \href{https://heobs.org}{https://heobs.org} into 4 classes, include:
\begin{description}
\item[Heritage] \hfill \\ A place of cultural, historical, or natural significance for a group or society.
\item[Beings] \hfill \\ Any form of life, such as a plant or a living creature, whether human or other animal.
\item[Scenery] \hfill \\ Any form of landscapes which show little or no human activity and are created in the pursuit of a pure, unsullied depiction of nature, also known as scenery.
\item[Other] \hfill \\ Any other type of image that doesn't represent a photograph, such as painting, illustration, any object.	
\end{description}

\section{State of the art}

\section{Fuctional Requirements}

\section{Non-functional Requirements}

\section{Prototype}

\bibliographystyle{ieeetr}
\bibliography{bibfile}
\end{document}